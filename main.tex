\documentclass[a4paper,12pt]{article}
\usepackage{cmap}
\usepackage[usenames,dvipsnames]{xcolor}
\usepackage{polyglossia}
\usepackage{xspace}
\setdefaultlanguage[spelling=modern]{russian}
\setotherlanguage{english}
\defaultfontfeatures{Ligatures={TeX}}
\setmainfont[Ligatures={TeX}]{CMU Serif}
\setsansfont{CMU Sans Serif}
\setmonofont{CMU Typewriter Text}

\usepackage{graphicx}
\usepackage{minted}
\usepackage{listings}
\usepackage{amsthm,amsmath,amssymb}

\usepackage[
  left=0.50in,
  right=0.50in,
  top=0.8in,
  bottom=0.7in,
  headheight=0.8in]{geometry}

\usepackage{fancyhdr}
%\pagenumbering{arabic}

\pagestyle{fancy}
\fancyhf{}
\renewcommand{\headrulewidth}{0pt}
\fancyhead[R]{\thepage}
\fancyhead[L]{SPb HSE (Alferov, Tukh, Yutman)}

%% \newcommand\enablecode{
%%   \usepackage{listings}
%%   \lstset{
%%     belowcaptionskip=1\baselineskip,
%%     breaklines=true,
%%     frame=L,
%%     xleftmargin=\parindent,
%%     showstringspaces=false,
%%     basicstyle=\footnotesize\ttfamily,
%%     keywordstyle=\bfseries\color{blue},
%%     commentstyle=\itshape\color{Maroon},
%%     identifierstyle=\color{black},
%%     stringstyle=\color{orange},
%%     numbers=left
%%   }

%%   % Some voodoo magic:
%%   % При желании можно адаптировать цвета под себя,
%%   % Для этого скопируйте это в conspect.tex с новым названием стиля и отредактируйте
%%   % Соответствующие куски.
%%   \lstdefinestyle{supercpp} {
%%     language=C++,
%%     deletekeywords={int, long, char, short, unsigned, signed,
%%       uint64\_t, int64\_t, uint32\_t, int32\_t, uint16\_t, int16\_t, uint8\_t, int8\_t,
%%       size\_t, ptrdiff\_t, \#include,\#define,\#if,\#ifdef,\#ifndef},
%%     classoffset=1,
%%     morekeywords={vector,stack,queue,set,map,unordered\_set,unordered\_map,deque,array,string,multiset,multimap,
%%       int, long, char, short, unsigned, signed,
%%       uint64\_t, int64\_t, uint32\_t, int32\_t, uint16\_t, int16\_t, uint8\_t, int8\_t,
%%       size\_t, ptrdiff\_t
%%     },
%%     keywordstyle=\bfseries\color{green!40!black},
%%     classoffset=0,
%%     classoffset=2,
%%     morekeywords={std},
%%     keywordstyle=\bfseries\color{ForestGreen},
%%     classoffset=0,
%%     morecomment=[l][\bfseries\color{purple!99!black}]{\#}
%%   }
%% }

\begin{document}

\topskip0pt
\vspace*{\fill}
\begin{center}{ \Large Team Reference} 
\end{center}
\vspace*{\fill}
\pagebreak

\textbf{Pollard} 
\inputminted[linenos, breaklines]{c++}{code/pollard.cpp}
\pagebreak
\textbf{pragma}
\begin{verbatim}
#pragma GCC optimize("O3,no-stack-protector")
#pragma GCC target(sse,sse2,sse4,ssse3,popcnt,abm,mmx,avx,tune=native)
\end{verbatim}

\textbf{Алгебра} 
{\bf Pick}\\
В + Г / 2 − 1 = AREA,\\
где В — количество целочисленных точек внутри многоугольника, а Г — количество целочисленных точек на границе многоугольника.\\\\

{\bf Newton}\\
$x_{i+1}=x_i-\frac {f(x_i)} {f'(x_i)}$\\\\

{\bf Catalan}\\
$C_n=\sum\limits_{k=0}^{n-1} C_{k} C_{n-1-k}$\\
$C_i=\frac 1 {n + 1} \binom {2n} {n}$\\\\

{\bf Кол-во графов}\\
$G_N:=2^{n(n-1)/2}$\\
Количество связных помеченных графов\\
$Conn_N = G_N - \frac 1 N \sum\limits_{K=1}^{N-1} K \binom N K Conn_K G_{N-K}$\\\\
Количество помеченных графов с K компонентами связности\\
$D[N][K]=\sum\limits_{S=1}^N \binom {N-1} {S-1} Conn_S D[N-S][K-1]$\\\\

{\bf Miller-Rabbin}
\begin{verbatim}
a=a^t
FOR i = 1...s
    if a^2=1 && |a|!=1
        NOT PRIME
    a=a^2
return a==1 ? PRIME : NOT PRIME
\end{verbatim}


{\bf Интегрирование по формуле Симпсона}\\
$\int_a^b f(x)dx ?$\\
$x_i := a+ih, i=0\ldots 2n$\\
$h = \frac {b-a} {2n}$\\

$\int = (f(x_0)+4f(x_1)+2f(x_2)+4f(x_3)+2f(x_4)+\ldots+4f(x_{2n-1})+f(x_{2n}))) \frac h 3$\\
$Погрешность имеет порядок уменьшения как O(n^4).$

{\bf Простые числа}\\
1009,1013;10007,10009;100003,100019\\
1000003,1000033;10000019,10000079\\
100000007,100000037\\
10000000019,10000000033\\
1000000000039,1000000000061\\
100000000000031,100000000000067\\
10000000000000061,10000000000000069\\
1000000000000000003,1000000000000000009\\

{\bf Числа для Фурье}\\
\begin{itemize}
\item prime: $7340033 = 7·2^{20} + 1; w: 5 (w^{2^{20}} = 1)$
\item prime: $13631489 = 13·2^{20} + 1; w: 3 (w^{2^{20}} = 1)$
\item prime: $23068673 = 11·2^{21} + 1; w: 38 (w^{2^{21}} = 1)$
\item prime: $69206017 = 33·2^{21} + 1; w: 45 (w^{2^{21}} = 1)$
\item prime: $81788929 = 39·2^{21} + 1; w: 94 (w^{2^{21}} = 1)$
\item prime: $104857601 = 25·2^{22} + 1; w: 21 (w^{2^{22}} = 1)$
\item prime: $113246209 = 27·2^{22} + 1; w: 66 (w^{2^{22}} = 1)$
\item prime: $138412033 = 33·2^{22} + 1; w: 30 (w^{2^{22}} = 1)$
\item prime: $167772161 = 5·2^{25} + 1; w: 17 (w^{2^{25}} = 1)$
\item prime: $469762049 = 7·2^{26} + 1; w: 30 (w^{2^{26}} = 1)$
\item prime: $998244353 = 7·17·2^{23} + 1; w: 3^{7 * 17}$.
\end{itemize}


\pagebreak
\inputminted[linenos, breaklines]{c++}{code/convex_hull_sqrt.cpp} \pagebreak
\inputminted[linenos, breaklines]{c++}{code/automaton.cpp} \pagebreak
\inputminted[linenos, breaklines]{c++}{code/automaton2.cpp} \pagebreak
\inputminted[linenos, breaklines]{c++}{code/discrete_log.cpp} \pagebreak
\inputminted[linenos, breaklines]{c++}{code/hungarian.cpp} \pagebreak
\inputminted[linenos, breaklines]{c++}{code/max_flow.cpp} \pagebreak
\inputminted[linenos, breaklines]{c++}{code/min_cost.cpp} \pagebreak
\inputminted[linenos, breaklines]{c++}{code/min_cyclic_shift.cpp}
\inputminted[linenos, breaklines]{c++}{code/sum_over_subsets.cpp}
 \pagebreak
\inputminted[linenos, breaklines]{c++}{code/strings.cpp} \pagebreak
\textbf{suf array + lcp}
\inputminted[linenos, breaklines]{c++}{code/suffmas-e-maxx.cpp} \pagebreak
\inputminted[linenos, breaklines]{c++}{code/kasai.cpp} \pagebreak
\pagebreak
\inputminted[linenos, breaklines]{c++}{code/prime_roots.cpp} \pagebreak
\inputminted[linenos, breaklines]{c++}{code/fft_sk.cpp}
\inputminted[linenos, breaklines]{c++}{code/fft_and_ds.cpp}
\pagebreak
{\textbf DM}\\

{\bf Кол-во корневых деревьев}:\\
$t(G)=\frac 1 n \lambda_2 \ldots \lambda_n$ ($\lambda_1=0$)\\

{\bf Кол-во эйлеровых циклов}:\\
$e(D)=t^-(D,x) \cdot \prod\limits_{y\in D} (outdeg(y)-1)!$\\

{\bf Наличие совершенного паросочетания}:\\
$T$ -- матрица с нулями на диагонали. Если есть ребро $(i, j)$, то $a_{i,j}:=x_{i,j}$, $a_{j,i}=-x_{i,j}$\\
$\det(T)=0 \Leftrightarrow$ нет совершенного паросочетания.\\
\end{document}
